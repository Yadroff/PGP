\section{Метод решения}
Для выделения границ изображения необходимо осуществить на каждом его пикселе найти градиент якрости. 
Для этого применяются две операции свертки, по одной на компоненту вектора градиента. 
При этом свертка осуществляется не по исходным значениям цвета, а по величине яркости, вычисляемой по формуле 
$u = 0.299r +  0.587g + 0.114b$. По полученным компонентам находится модуль градиента, его значение необходимо ограничить 
максимальной величиной компоненты цвета -- 255. 
В пиксель выходного изображения записывается модуль градиента во все цветовые каналы, альфа канал не изменяется. 
Сложность алгоритма --- $O(w \cdot h)$.

\section{Описание программы}
Проект я разделил на следущие модули:
\begin{enumerate}
    \item $common \; defines$ - содержит множество полезных дефайнов для более приятной работы с $CUDA$ и $C++$.
    \item $common \; structures$ - содержит класс-обертку над $CUDA$ событиями и класс-обертку над $std\!::\!stringstream$, позволяющим конструировать строки ($sprintf$ с возможностями $C++$).
    Также сюда были помещены классы-обертки над текстурными объектами и ресурсами.
    \item $operation \; system$ - содержит операции, связанные с системными вызовами (например, получение информации о количестве оперативной памяти). Реализован только для ОС $Windows$.
    \item $kernel$ - непосредственно программа, выполняющия все операции, связанные с вводом, вычислениями и выводом.
\end{enumerate}

Функция ядра осуществляет описанный выше алгоритм и принимает четыре аргумента: объект текстуры, указатель на массив входных данных и размеры изображения.
Ядра свертки записаны в константную память, что позволяет кэшировать их и тем самым обеспечивать быстрый доступ к ним.

Также для замера времени была написана программа на языке программирования $Python$, состоящая из слеудющих модулей:
\begin{enumerate}
    \item $build$ используется для компиляции $Microsoft\:Visual\:Studio$ проекта (файла с расширением $.sln$).
    \item $tests \; generator$ используется для генерации тестов и ответов на них (ответы генерируются с помощью программы, написанной под $CPU$).
    \item $cheker$ используется для проверки работоспособности программы.
     Для этого запускается раннее сгенерированный $.exe$ файл с входными данными из теста, далее вывод программмы проверяется с ответом.
    \item $benchmark$ используется для замера скорости работы программы. Для этого программа запускается 5 раз с входными тестовыми данными.
     Для каждого теста временем работы будет минимальное время работы из всех запусков данного теста.
    \item $converter$ используется для конвертации изображения в бинарный вид и для получения изображения из бинарных данных.
    \item $main$ используется в качестве обертке над описанными выше модулями для более комфортного запуска модулей.
\end{enumerate}