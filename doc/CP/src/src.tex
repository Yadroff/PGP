\section{Метод решения}
В курсовой работе реализован алгоритм обратной трассировки лучей с фиксированным числом переотражений и без явного использования рекурсии. Основная единица исполнения трассировки лучей --- трассировка одного луча (вычисление точки его пересечения со сценой), вычисления цвета в данной точке и создание отраженных или преломленных лучей. Исходные и новые лучи можно хранить в двух массивах и таким образом, одна итерация трассировки всех лучей соответствует одному рекурсивному спуску.

Для вычисления цвета в точке пересечения используется затенение по Фонгу, согласно которому он складывается из цветовых интенсивностей трех компонент освещения: фоновой, рассеянной и бликовой
$$I = I_a + I_d + I_s$$

Фоновая компонента:
$$I_a = K_a i_a$$
где $K_a$ --- способность материала воспринимать фоновое освещение, $i_a$ --- интенсивность фонового освещения.

Рассеянная компонента:
$$I_d = K_d \cos(\vec{L}, \vec{N}) i_d = K_d (\vec{L} \cdot \vec{N}) i_d$$
где $K_d$ --- способность материала воспринимать рассеянное освещение, $i_a$ --- интенсивность рассеянного освещения, $\vec{L}$ --- направление из точки на источник света, $\vec{N}$ --- нормаль в точке.

Бликовая (зеркальная) компонента:
$$I_s = K_s \cos^p(\vec{R}, \vec{V}) i_s = K_s (\vec{R} \cdot \vec{V})^p i_s$$
где $K_s$ --- способность материала воспринимать бликовое освещение, $p$ --- коэффициент блеска, $i_a$ --- интенсивность бликового освещения, $\vec{R}$ --- направление отраженного луча, $\vec{V}$ --- направление из точки на наблюдателя.

Для создания теней интенсивность источника света в точке корректируется с учетом коэффицента пропускания и цвета материалов, через которые проходит луч на пути к этой точке.

В качестве алгоритма сглаживания используется SSAA --- изображение рендерится в $N$ раз большем размере, итоговое изображение получается усреднением цвета в непересекающихся окнах $N \times N$.


\section{Описание программы}
Проект я разделил на следущие модули:
\begin{enumerate}
    \item $common$ - содержит множество полезных дефайнов, структур данных для более приятной работы с $CUDA$ и $C++$
    \item $Math$ - содержит классы векторов (математических), матриц и операций над ними
    \item $Engine$ - содержит в себе класс объекта, сцены, камеры.
    \item $Render$ - содержит в себе классы рендереров, отвечающих за непосредественно рендер сцены
\end{enumerate}

Также для замера времени была написана программа на языке программирования $Python$, состоящая из слеудющих модулей:
\begin{enumerate}
    \item $build$ используется для компиляции $Microsoft\:Visual\:Studio$ проекта (файла с расширением $.sln$).
    \item $converter$ используется для конвертации изображения в бинарный вид и для получения изображения из бинарных данных.
    \item $video$ используется для конвертации изображений в $.gif$
    \item $main$ используется в качестве обертке над описанными выше модулями для более комфортного запуска модулей.
\end{enumerate}